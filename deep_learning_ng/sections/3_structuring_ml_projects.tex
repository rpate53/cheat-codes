\graphicspath{{./images/}}
\titleformat{\section}
    {\normalfont\fontfamily{phv}\fontsize{9}{0}\bfseries}{\thesection}{1em}{}
    \section{Structuring Machine Learning Projects}
\renewcommand\labelitemi{{\boldmath$\cdot$}}
\setlist{nolistsep}
\begin{itemize}[noitemsep]
    \item Having a \textbf{single number evaluation metric} can improve decision making (i.e. F1 score for precision and recall)
    \item If there are multiple things, pick one \textbf{optimizing metric subject to a satisfying metric} (i.e. maximize accuracy subject to $<=$ false positives)
    \item If dev and test set are from different distributions, consider shuffling them together and partitioning into both sets (i.e. otherwise throwing darts at a bullseye that's now in another location). It's okay for the training and dev distributions to be different.
    \item In the era of big data, the old rule of thumb of 70/30 for train/test no longer applies. The trend has been to \textbf{use more data for training and less for dev and test sets}.
    \item \textbf{Bayes optimal error - the theoretical best possible error} - it is unachievable. Human error is a proxy for Bayes error. Compare human-level error to training error to determine if you need to reduce bias. \\
    Avoidable bias = Training Error - Human Error.

    \includegraphics[height=0.4\linewidth]{avoidable_bias}

    \item Error Analysis - Look at misclassifications to understand where most of the misclassifications are coming from so you can focus your attention on that
    \item \textbf{Build your first system quickly and then iterate}
    \begin{itemize}
        \item Set up dev/test set and metric
        \item Build initial system quickly
        \item Use Bias/Variance and Erorr Analysis to prioritize next steps
    \end{itemize}
    \item Data Mismatch - Algorithm does well on training data but not on dev data from a different distribution.
    \begin{itemize}
        \item Do manual error analysis to understand difference between training and dev/test sets
        \item Generate more training data similar to dev sets
    \end{itemize}
    \includegraphics[height=0.4\linewidth]{error_analysis_tradeoff}

    \item Transfer Learning
    \begin{itemize}
        \item Remove last layer and its weights and create new layer(s) relevant to new problem. Retrain the last few layer(s) or all if you have enough data.
        \item Pre-training - initial training phase where model learns general features
        \item Fine-tuning - updating weights on a pre-training model (either all or a subset of its layers)
        \item When to use?
        \begin{itemize}
            \item Task A and Task B have the same input x
            \item A lot more data for Task A than Task B
            \item Low level features from A could be helpful for learning B
        \end{itemize}
    \end{itemize}
    \item Multi-task learning 
    \begin{itemize}
        \item Allows you to train one neural network to do many tasks and can give better performance than if you were to do the tasks in isolation
        \item $\mathcal{L}(\hat{y_{j}}^{(i)},y_{j}^{(i)}) = -y_{j}^{(i)}log\hat{y_{j}}^{(i)}-(1-y_{j}^{(i)})log(1-\hat{y_{j}}^{(i)})$ \\
        $Cost = \frac{1}{m}\sum_{i=1}^{m}\sum_{j=1}^{C}\mathcal{L}(\hat{y_{j}}^{(i)},y_{j}^{(i)})$
        \item When to use?
        \begin{itemize}
            \item Training on a \textbf{set of tasks} that could benefit from having shared lower-level features
            \item Usually amount of data you have for each task is quite similar
            \item Can training a big enough neural network to do well on all of the tasks
        \end{itemize}
    \end{itemize}
    \item Transfer Learning is used a lot more than Multi-task Learning (computer vision is the one major exception)
    \item End-to-end Deep Learning
    \begin{itemize}
        \item Several data processing systems require multiple stages of processing. End-to-end deep learning takes all of those stages and replaces it with a single neural network. \textbf{Not always useful}
        \item Not work well \\
        - Face recognition \\
        \includegraphics[height=0.5\linewidth]{end-to-end_face_recognition}

        - Image of skeleton to predict age of person \\
        \includegraphics[height=0.3\linewidth]{skeleton_to_age}

        - Autonomous driving \\ 
        \includegraphics[height=0.13\linewidth]{end-to-end_autnomous_driving}

        \item Works well - machine translation
        \includegraphics[height=0.13\linewidth]{end-to-end_machine_translation}
    
        \item When to use?
        \begin{itemize}
            \item Pros: \\
            - Let the data speak \\
            - Less hand-designing of components needed 
            \item Cons: \\
            - May need large amount of data \\ 
            - Excludes potentially useful hand-designed components
        \end{itemize}
    \end{itemize}
\end{itemize}