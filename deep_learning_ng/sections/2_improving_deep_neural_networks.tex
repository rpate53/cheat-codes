\graphicspath{{./images/}}
\titleformat{\section}
    {\normalfont\fontfamily{phv}\fontsize{9}{0}\bfseries}{\thesection}{1em}{}
    \section{Hyperparameter tuning, regularization, and optimization}
\renewcommand\labelitemi{{\boldmath$\cdot$}}
\setlist{nolistsep}
\begin{itemize}[noitemsep]
    \item Train/dev/test sets 
    \begin{itemize}
        \item In the age of big data, it's fine if you don't comply with the rule of thumb of 70/30 (train/test split)
        \item Make sure dev and test set come from same distribution
        \item Not having a test set might be okay (only dev set)
    \end{itemize}
    \item Bias/Variance
    \begin{itemize}
        \item Bias - Variance used to assess how to best improve the model. You can assess them by examining the train/dev set errors and comparing them to the human (or bayes error).
        \item High Bias - a non-complex model that underfits the training data. Create bigger network, train longer to lower it.
        \item High Variance - a complex model that overfits the training data. Add more data and regularize to lower it.
    \end{itemize}
    \item Regularization
    \begin{itemize}
        \item Generally, the first step to fixing high variance. Adding more data isn't always possible.
        \item 1. Frobenius Norm
        \begin{itemize}
            \item The Frobenius Norm penalty will penalize the weight matrices from being too large
            \item Frobenius Norm: \\ $||w^{[l]}||^2_F=\Sigma_{i=1}^{n^{[l]}}\Sigma_{j=1}^{n^{[l-1]}}(w_{ij}^{[l]})^2$
            \item $J(w,b) = \frac{1}{m}\Sigma_{i=1}^{m}L(\hat{y}^{(i)},y^{(i)})+\frac{\lambda}{2m}\Sigma_{l=1}^{L}||w^{[l]}||^{2}_F$
            \item Weight decay - regularization technique that results in gradient descent shrinking the weights on every iteration
            \item $\uparrow \lambda \Rightarrow \downarrow w^{[l]} \Rightarrow \downarrow z^{[l]} \Rightarrow$ every layer of the network is linear
            
            \includegraphics[height=0.4\linewidth]{regularization_tanh}

        \end{itemize}
        \item 2. Dropout
        \begin{itemize}
            \item Randomly shutting down neurons in each iteration - mostly applied in computer vision problems
            \item Inverted dropout - divide the activations by the keep probability to ensure the result of the cost will have the same expected value without drop-out.
            \item It works by spreading out the weights and preventing reliance on any one feature.
            \item Dropout is only used in training.
            \item Dropout is applied during forward and backward propagation
        \end{itemize}
        \item 3. Data augmentation
        \begin{itemize}
            \item An inexpensive way of generating more data to reduce overfitting (i.e. invert, distort, crop, .... to images)
        \end{itemize}
        \item 4. Early stopping
        \begin{itemize}
            \item Stop training a neural network once the dev set error starts to increase.
            
            \includegraphics[height=0.5\linewidth]{early_stopping}
            \item Con: not using orthogonalization which de-couples (1) optimizing the cost function and (2) not overfitting the model
        \end{itemize}
    \end{itemize}
    \item Optimization
    \begin{itemize}
        \item Normalization
        \begin{itemize}
            \item Technique of putting all values on the same scale. Generally by taking the input, subtracting the mean, and dividing by the standard deviation.
            \item Cost function will be more round and faster to optimize
            
            \includegraphics[height=0.2\linewidth]{normalization}
        \end{itemize}
        \item Vanishing/Exploding Gradients
        \begin{itemize}
            \item When parameters are all slightly larger than one $->$ Exploding Gradient
            \item When parameters are all slightly smaller than one $->$ Vanishing Gradient
            \item Common in deep networks - both will prevent the model from converging to the optimal set of parameters
            \item Xavier initialization solution - initialize the weight matrix of each layer such that its variance is $1/n$ where $n$ is the number of input features. \\
            Relu: $\sqrt{\frac{2}{n^{[l-1]}}}$ \\
            Tanh: $\sqrt{\frac{1}{n^{[l-1]}}}$
        \end{itemize}
        \item Gradient checking
        \begin{itemize}
            \item Gradient checking verify if you computed the gradients properly. It checks the closeness between the gradients (via backpropagation) and the gradients (via numerical approximation using forward propagation).
            \item Only use it to make sure your code is correct.
            \item $gradapprox = \frac{\partial J}{\partial \theta}=\lim_{\epsilon \to 0} \frac{J(\theta + \epsilon)-J(\theta - \epsilon)}{2 \epsilon}$ \\
            $difference = \frac{||grad - gradapprox||_{2}}{||grad||_{2}+||gradapprox||_{2}}$ \\
            If $difference < 2*10^{-7}$, then your backward propagation works fine.
        \end{itemize}
    \end{itemize}
    \item Optimization Algorithms
    \begin{itemize}
        \item Mini-batch - partition examples into mini batches and use them to compute the parameters. Get advantage of vectorization and faster convergence
        \item Stochastic gradient descent - special case where mini-batch size = 1
        \item Epoch - one pass through the training set
        \item For large datasets, mini-batch runs faster than batch gradient descent
        \item Ideal size
        \begin{itemize}
            \item if $<=$ 2,000 examples, use batch gradient descent
            \item mini-batch sizes are a power of 2 (i.e. 64, 128, 256, ...) to take advantage of how ram is layed out in hardware
            \item make sure mini-batch size fits in cpu/gpu memory
        \end{itemize}
        \item Implementation \\
            \begin{lstlisting}
                for t = 1,...,num_batches:
                    forward prop on $X^{t}$ (vectorized implementation)
                        $Z^{[1]} = W^{[1]}X^{t}+b^{1}$
                        $A^{[1]} = g^{[1]}(Z^{[1]})$
                        $.$
                        $.$
                        $A^{[L]} = g^{[L]}(Z^{[L]})$
                    compute cost 
                        $J^{t}=\frac{1}{batchsize}\sum_{i=1}^{l}\mathcal{L}(\hat{y}^{(i)},y^{(i)})+\frac{\lambda}{2*batchsize}\sum_{i=1}^{l}||W^{[l]}||^{2}_{F}$
                    back prop to compute gradients
                    update parameters
                        $W^{[l]}:=W^{[l]}-\alpha\partial W^{[l]}$
                        $b^{[l]}:=b^{[l]}-\alpha\partial b^{[l]}$   
            \end{lstlisting}
        \item Gradient descent with momentum
        \begin{itemize}
            \item Exponentially weighted average (momentum)
            \begin{itemize}
                \item $\beta$ is our smoothening parameter (i.e. $\beta$ = 0.9 is averaging our loss over 10 gradients)
                \item But this works bad for initial few examples. You can do better if you divide by $1-\beta^{t}$
                \item $v_{t} = \beta*v_{t-1}+(1-\beta)\theta_{t}$
                \item $v_{dW^{[l]}}=\beta*v_{dW^{[l]}}+(1-\beta)dW^{[l]}$ \\
                $W^{[l]} := W^{[l]}-\alpha*v_{dW^{[l]}}$
                \item $v_{db^{[l]}}=\beta*v_{db^{[l]}}+(1-\beta)db^{[l]}$ \\
                $b^{[l]} := b^{[l]}-\alpha*v_{db^{[l]}}$
            \end{itemize}
            \item RMS Prop 
            \begin{itemize}
                \item Adapt learning rate based on moving average of squared gradients
                \item $S_{dW}=\beta*S_{dW} + (1-\beta)dW^{2}$ \\
                $S_{db}=\beta*S_{db} + (1-\beta)db^{2}$ \\
                $W := W-\alpha\frac{\partial dW}{\sqrt{S_{dW}}}$ \\
                $b := b-\alpha\frac{\partial db}{\sqrt{S_{db}}}$
            \end{itemize}
            \item Adam 
            \begin{itemize}
                \item Combines momentum (moving average of the first gradient) with RMS prop (moving average of the squared gradient) 
                \includegraphics[height=0.5\linewidth]{adam-algorithm}

            \end{itemize}
            \item Learning rate decay - lower the learning rate as you get closer to the minima. One variation is below \\
            $\alpha = \frac{1}{1+decayrate*epochnum}*\alpha_{0}$
            \item Learning rate decay controls the step size while momentum controls the direction of the steps taken by the optimizer
            \includegraphics[height=0.15\linewidth]{optima_graph}

            \item In a high dimensional space, most optima are saddle points, so you will not get stuck in local optima. However, plateaus are a problem and the gradient may become very close to 0 and learning will be very slow. This is where momentum, rms prop, or adam can help.
        \end{itemize}
        \item Hyperparameter tuning
        \begin{itemize}
            \item Try random values: don't use a grid. Focus more resources on searching in a fine focused region.
            \item Appropriately scale for hyperparameters (i.e. scale is wide, consider log scale). This is especially true for exponentially weighted averages (i.e. $\beta$)
        \end{itemize}
        \item Batch normalization
        \begin{itemize}
            \item Normalizing the units in the hidden layers to speed up the learning
            \item Want to avoid normalizing hidden units to have $\mu=0$ and $\sigma=1$ so that model can learn non-linearity. Ensure hidden units have standardized mean and variance controlled by $\gamma$ and $\beta$ \\
            \includegraphics[height=0.5\linewidth]{batch_norm}

            \item Slight regularization effect: makes units in later hidden units more robust to changes in earlier input and/or hidden layers.
            \item What $\mu$ and $\sigma$ do we use for testing? Both parameters are estimated using an exponentially weighted average across mini-batches used during training.
        \end{itemize}
        \item Softmax regression
        \begin{itemize}
            \item Generalizes logistic regression to multiple classes
            \includegraphics[height=0.2\linewidth]{softmax_nn}

            \item Activation function: \\
            $t = e^{z^{[L]}}$ \\
            $a^{[L]} = \frac{e^{z^{[L]}}}{\sum_{j=1}^{c}t_{j}}$
            \item Loss: $\mathcal{L}(\hat{y},y) = -\sum_{j=1}^{c}y_{j}log\hat{y}_{j}$
            \item Cost: $J(w^{[1]},b^{[1]},...)=\frac{1}{m}\sum_{i=1}^{m}\mathcal{L}(\hat{y}^{(i)},y^{(i)})$
        \end{itemize}
    \end{itemize}
\end{itemize}