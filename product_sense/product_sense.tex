\documentclass{article}
\usepackage{bm,sectsty,fancyhdr,multicol,lastpage}
\usepackage{amsmath,enumitem,tabularx,textcomp,nccmath,amssymb}
\usepackage{listings} % Insert bullet points
\usepackage[a4paper,left=1cm,right=1cm,top=2cm,bottom=2cm,headsep=0.5cm]{geometry}
\usepackage{xcolor} % Format color in code cells
\usepackage{titlesec} % Format title section
\usepackage{graphicx} % Embed images
\usepackage[document]{ragged2e} % Fix uneven spacing caused in multi col mode

% Define Colors
\definecolor{codegreen}{rgb}{0,0.6,0}
\definecolor{codegray}{rgb}{0.5,0.5,0.5}
\definecolor{codeorange}{rgb}{1,0.49,0}
\definecolor{backcolour}{rgb}{0.95,0.95,0.96}
\lstdefinestyle{mystyle}{
    backgroundcolor=\color{backcolour},
    commentstyle=\color{codegray},
    keywordstyle=\color{codeorange},
    numberstyle=\tiny\color{codegray},
    stringstyle=\color{codegreen},
    %basicstyle=\ttfamily\footnotesize,
    basicstyle=\tiny,
    breakatwhitespace=false,
    breaklines=true,
    captionpos=b,
    keepspaces=true,
    numbers=left,
    numbersep=5pt,
    showspaces=false,
    showstringspaces=false,
    showtabs=false,
    tabsize=2,
    xleftmargin=0pt,
    language=SQL,
    label={lst:code},
    mathescape=true,
    gobble=12
}
\lstset{style=mystyle}

% Header and Footer
\pagestyle{fancy}
\fancyhf{}
\fancyhead[L]{Product Sense Cheatsheet}
\fancyfoot[R]{Riten Patel}
\fancyfoot[L]{\LaTeX}
\fancyfoot[C]{\thepage\ of \pageref{LastPage}}
\renewcommand{\headrulewidth}{1.4pt}
\renewcommand{\footrulewidth}{1.4pt}

% Configurations
\sectionfont{\large}
\setlength{\columnsep}{0.5cm}
\setlength{\columnseprule}{1.4pt}
\setlength{\parindent}{0pt}
\setlength{\parskip}{-6pt}
\newcolumntype{Y}{>{\centering\arraybackslash}X}
\setlist[itemize]{nolistsep,left=0pt}
\graphicspath{{./images/}}
\tolerance=9999
\emergencystretch=10pt
\hyphenpenalty=10000
\exhyphenpenalty=100


\begin{document}
\begin{multicols*}{2}
    \raggedcolumns

    % Introduction
    \titleformat{\section}
        {\normalfont\fontfamily{phv}\fontsize{10}{0}\bfseries}{\thesection}{1em}{}
    \section{Types of Product Interview Questions}
    \renewcommand\labelitemi{{\boldmath$\cdot$}}
    \begin{itemize}[noitemsep]
        \item Defining a product metric
        \begin{itemize}
            \item What metrics would you analyze to validate
            changes to an existing feature?
            \item What metrics would you analyze to validate their 
            hypothesis?
        \end{itemize}
        \item Diagnosing a metric change 
        \begin{itemize}
            \item How would you investigate the root cause behind
            a metric going up or down?
            \item What if other metrics changed at the same time?
        \end{itemize}
        \item Barinstorming product features
        \begin{itemize}
            \item Should a company launch a new product?
            \item What feature ideas would improve a certain 
            metric?
        \end{itemize}
        \item Designing A/B tests
        \begin{itemize}
            \item How would you set up an A/B test to measure
             the success of a new feature?
            \item What are some likely pitfalls you might run 
            into while performing A/B tests and how would 
            you deal with them?
        \end{itemize}
    \end{itemize}

    %Big-Picture Advice
    \titleformat{\section}
        {\normalfont\fontfamily{phv}\fontsize{10}{0}\bfseries}{\thesection}{1em}{}
    \section{Big-Picture Advice}
    \renewcommand\labelitemi{{\boldmath$\cdot$}}
    \begin{itemize}[noitemsep]
        \item Ask clarifying questions
        \begin{itemize}
            \item Who is the end-user?
            \item Who is the stakeholder?
            \item What is our goal?
        \end{itemize}
        \item Establish problem boundaries
        \begin{itemize}
            \item Inform interviewer what you're purposely 
            choosing to ignore to solve the problem in the 
            time allotted.
        \end{itemize}
        \item Talk Out Loud
        \item Be conversational
        \begin{itemize}
            \item Am I on the right track?
        \end{itemize}
        \item Keep goals forefront
        \item Bring in outside experience tactfully
    \end{itemize}

    %How to develop your product sense
    \titleformat{\section}
        {\normalfont\fontfamily{phv}\fontsize{10}{0}\bfseries}{\thesection}{1em}{}
    \section{How to develop your product sense}
    \renewcommand\labelitemi{{\boldmath$\cdot$}}
    \begin{itemize}[noitemsep]
        \item Create a daily habit
        \begin{itemize}
            \item Who was the product created for?
            \item What's the main problem it was designed to solve?
            \item What are the product's end-user benefits?
            \item How do the design and marketing convey the product's
            purpose and benefits?
            \item How does the produt tie in with the company's 
            mission and vision?
            \item i.e. Snapchat vs. Instagram 
            (communication vs. consumption)
        \end{itemize}
        \item Analyze the reviews and calibrate
        \begin{itemize}
            \item Reddit to see unfiltered conversations
            \item Apps: The App Store and Google Play
            \item Enterprise Products: G2 Crowd and 
            Gartner Special Reports
            \item Physical: Amazon reviews
        \end{itemize}
    \end{itemize}

    %How to develop your business sense
    \titleformat{\section}
        {\normalfont\fontfamily{phv}\fontsize{10}{0}\bfseries}{\thesection}{1em}{}
    \section{How to develop your business sense}
    \renewcommand\labelitemi{{\boldmath$\cdot$}}
    \begin{itemize}[noitemsep]
        \item Business Model
        \begin{itemize}
            \item How does the business monetize? What product 
            levers can be pulled to improve the 
            business' ability to monetize?
        \end{itemize}
        \item Metrics 
        \begin{itemize}
            \item Which key performance indicators (KPIs) would 
            I measure? What factors/variables influence those 
            metrics?
        \end{itemize}
        \item Landscape
        \begin{itemize}
            \item Who are the competitors?
            \item Who are the partners?
        \end{itemize}
    \end{itemize}

    %How to develop your domain experience
    \titleformat{\section}
        {\normalfont\fontfamily{phv}\fontsize{10}{0}\bfseries}{\thesection}{1em}{}
    \section{How to develop your domain experience}
    \renewcommand\labelitemi{{\boldmath$\cdot$}}
    \begin{itemize}[noitemsep]
        \item I.E. you have an Uber Eats Interview coming up.
        \begin{itemize}
            \item Learn how Uber makes money and how much 
            of it comes from their transportation products versus
            delivery business
            \item How does Uber Eats fit into Uber's overall 
            strategy?
            \item What are the key inputs for Uber's pricing 
            and payout algorithms that determine how much 
            it charges a customer and how much it pays the 
            delivery driver and restaurant?
        \end{itemize}
        \item If public, look at earnings reports/note business metrics
        \item If private, look at comparable companies.
        \item Google search "company name business model"
        \item Look at Company's Engineering Blog
        \item Use the product!! Smaller companies will test you.
        \item Useful in asking interviewer questions at the end. i.e. 
        "I was reading about the food delivery time estimation algorithm 
        on the blog and found X fascinating. I was curious why you used 
        approach Y, and if you ever thought about trying out Z instead?"
    \end{itemize}

    %Metrics for Product and Case Interviews
    \titleformat{\section}
        {\normalfont\fontfamily{phv}\fontsize{10}{0}\bfseries}{\thesection}{1em}{}
    \section{Metrics to use}
    \renewcommand\labelitemi{{\boldmath$\cdot$}}
    \begin{itemize}[noitemsep]
        \item \textbf{A}cquisition metrics
        \begin{itemize}
            \item How are people finding out about my product?
            \item new user counts 
            \item sign-up conversion rates 
            \item customer acquisition costs (CAC)
        \end{itemize}
        \item \textbf{A}ctivation metrics 
        \begin{itemize}
            \item Do users have a great first experience
            \item number of people who make first delivery order
            \item number of users who view more than 10 unique posts
        \end{itemize}
        \item User Engagement
        \begin{itemize}
            \item Count unique number of users who took a core action
            \item DAU (Daily Active Users)
            \item WAU (Weekly Active Users)
            \item MAU (Monthly Active Users)
        \end{itemize}
        \item \textbf{Retention} metrics
        \begin{itemize}
            \item Do users come back?
            \item Churn - when users join and then leave permanently
            \item Maximize retention and minimize churn
            \item It's a lot more expensive to find a new customer than 
            retain an existing customer.
            \item Monthly retention
            \item Monthly churn
        \end{itemize}
        \item \textbf{Referral}
        \begin{itemize}
            \item Where a user shares a product with others
            \item k-factor = of the number of referrals, how many converted.
            Want a k-factor $>$ 1.
        \end{itemize}
        \item \textbf{Revenue}
        \begin{itemize}
            \item Lifetime value per customer (LTV) - amount 
            of money a customer brings into business before 
            they churn.
            \item Number of paid memberships.
            \item Revenue from ad impressions.
            \item Initially shouldn't be primary focus. But a 
            byproduct of a nailing product-market-fit (i.e. users 
            are not going click on ads if Facebook is not great 
            at making connections)
        \end{itemize}
    \end{itemize}

    % What makes a metric good or bad?
    \titleformat{\section}
        {\normalfont\fontfamily{phv}\fontsize{10}{0}\bfseries}{\thesection}{1em}{}
    \section{What makes a metric good or bad}
    \renewcommand\labelitemi{{\boldmath$\cdot$}}
    \begin{itemize}[noitemsep]
        \item Data Scientists help Product Managers define apporpriate
        metrics
        \item Have the end result in mind before building a product
        \item \textbf{Examples of bad metrics}
        \begin{itemize}
            \item Vanity Metrics: The number of dating profiles viewed
            \item Irrelevant Metrics: Time spent using Facebook Dating.
            This metric is better for media consumption products like 
            YoutTube and Netflix, not activity-driven dating apps.
            \item Impractical Metrics: The number of 3rd dates that occured.
            More advanced dating apps have a "did you meet?" user 
            prompt or they use NLP on the conversation to determine 
            if they think you met, but this likely works only for a
            first date, after which the conversation moves off-app.
            \item Complicated Metrics: Is the metric easy to explain to 
            stakeholders?
            \item Delayed Metrics: Number of marriages that occurred. This is 
            impractical to know but would take a long time to figure out.
        \end{itemize}
        \item \textbf{Examples of good metrics}
        \begin{itemize}
            \item Meaningful: Tied to business goals
            \item Measurable: Simple to reliably track
            \item Understandable: Easy for stakeholders to understand.
            \item Timely: Can be collected with in a timeframe.
        \end{itemize}
        \item Important to define guardrail metrics. These are metrics 
        which shouldn't degrade as you're boosting the primary metric 
        (i.e. metric is to reduce the number of harmful posts but 
        don't want to remove everything. Still need to monitor 
        posts made, posts viewed, numbre of likes/comments)
    \end{itemize}

    %Product Metric Definition Question Strategy
    \titleformat{\section}
        {\normalfont\fontfamily{phv}\fontsize{10}{0}\bfseries}{\thesection}{1em}{}
    \section{Product Metric Question Strategy}
    \renewcommand\labelitemi{{\boldmath$\cdot$}}
    \begin{itemize}[noitemsep]
        \item 1. Clarify the product and its purpose
        \item 2. Explain the product and business goals. Tie it 
        back to the company's mission
        \item 3. Define Success metrics
    \end{itemize}

    %4-Step Framework for Diagnosing Metric Changes
    \titleformat{\section}
        {\normalfont\fontfamily{phv}\fontsize{10}{0}\bfseries}{\thesection}{1em}{}
    \section{Product Metric Question Strategy}
    \renewcommand\labelitemi{{\boldmath$\cdot$}}
    \begin{itemize}[noitemsep]
        \item 1. Scope Out the Metric Change
        \begin{itemize}
            \item Did numerator or denominator change?
            \item Is the change important?
            \item How often does it change?
            \item How big of a change is it?
        \end{itemize}
        \item 2. Hypothesize Contributing factors
        \begin{itemize}
            \item Accidental Changes
            \item Natural Change (seaonality, holidays, ...)
            \item Internal Changes (bug fixes, new features, ...)
            \item External Changes (Competitors launching new products, 
            pandemic, recession, ...)
        \end{itemize}
        \item 3. Validate Each factor
        \item 4. Classify Each factor
        \begin{itemize}
            \item Root cause: of the change 
            \item Contributing factor: not root cause but a factor
            \item Correlated result: symptom of root cause but not a factor 
            \item Unrelated factor: unrelated to metric change 
            \item Example: Why are comments per Instagram posts declining?
            Cohort Analysis $->$ Affecting Only New Posts $->$ PM reveals 
            new feature added giving users option to turn off comments 
            on post $->$ Remove these posts $->$ No change in comments per post $->$
            Root cause discovered!
        \end{itemize}
    \end{itemize}

    %A/B Testing and Experimental Design
    \titleformat{\section}
        {\normalfont\fontfamily{phv}\fontsize{10}{0}\bfseries}{\thesection}{1em}{}
    \section{A/B Testing and Experimental Design}
    \renewcommand\labelitemi{{\boldmath$\cdot$}}
    \begin{itemize}[noitemsep]
        \item 1. Pick a Metric to Test 
        \item 2. Define Thresholds
        \begin{itemize}
            \item Set $\alpha$
            \item Set $power = 1-\beta$
        \end{itemize}
        \item 3. Pick a sample size and experiment length
        (i.e. typical to run test for at least two weeks)
        \item 4. Assign Groups
        \begin{itemize}
            \item Make sure we randomize groups to avoid 
            confouding variables down the line 
        \end{itemize}
        \item When Not to A/B Test 
        \begin{itemize}
            \item Lack of infrastructure
            \item Lack of impact 
            \item Lack of traffic
            \item Lack of conviction
            \item Lack of isolation
        \end{itemize}
        \item Dealing with Non-Normality
        \begin{itemize}
            \item Bootstrapping
            \item Run alternate tests 
            \item Gathering more data
        \end{itemize}
        \item Dealing with multiple tests simultaneously. If you 
        run 100 A/B tests, you will have some succeed. Why? 
        How do you solve it?
        \begin{itemize}
            \item Bonferonni Correction - adjust for the 
            significance level required based on the total 
            number of tests running. $\frac{\alpha}{num \: tests}$
            \item FDR - false discovery rate. $\frac{FP}{TP+FP}$. 
            The rate of Type I errors.
            \item Dig into the experiment and see if anything 
            could have impacted the primary experiment.
        \end{itemize}
        \item Dealing with Network Effects - interferance between the 
        control and treatment groups (i.e. Facebook Live)
        \item Dealing with Novelty Effects - Social Media/PR 
        hypes up a feature and skews metrics
        \item Nuances: how does A/B test effect the guardrail metrics
        \item Launch with A/B Test Holdouts - easy to compare live
    \end{itemize}
\end{multicols*}
\end{document}